\documentclass[12pt]{article}

\usepackage[utf8]{inputenc}
\usepackage{graphicx}
\usepackage{amsmath, amssymb}
\usepackage[colorlinks=true, linkcolor=blue, citecolor=blue, urlcolor=blue]{hyperref}
\usepackage{caption}
\usepackage{subcaption}
\usepackage{geometry}
\usepackage{float}
\usepackage[backend=biber,style=numeric]{biblatex}
\addbibresource{./ref.bib}

\geometry{
    a4paper,
    left=30mm,
    right=25mm,
    top=25mm,
    bottom=25mm
}

\title{PANORMA: Pancreatis Cancer Diagnosis: Radiologist Meets AI}
% add subtitle

\author{Adem Kaya, George Lalidis, and Tam Van}

\date{May 2025}

\begin{document}

\maketitle

\section{Introduction}

\section{Methodology}

\subsection{Data Augmentation}
\subsubsection{Data Balancing}
\subsubsection{Data Alteration}
%\subsubsection{Data Masking}

\subsection{Custom Loss Functions}
\subsubsection{Dice and CE}
We implemented a custom compound loss that takes a weighted combination of the Dice loss and the Cross-Entropy 
(CE) loss. Dice loss is particularly popular for image segmentation tasks due to its robust nature even when applied
imbalanced datasets. 

CE is another commonly used metric that is used as a loss function for deep learning tasks. CE measures the difference
measures the difference between two probability distributions for some random variable and can be formally denoted:

\begin{equation}
\mathcal{L}_{\mathrm{CrossEntropy}}(\mathbf{Y}, \hat{\mathbf{Y}}) = -w_{y_n} \log\frac{\exp(\mathbf{y}_{n, \hat{\mathbf{y}}})}{\sum_{c=1}^{C} \exp(\mathbf{y}_{n,c})} \cdot \hat{\mathbf{y}}_n,
\end{equation}
% https://docs.pytorch.org/docs/stable/generated/torch.nn.CrossEntropyLoss.html
where 

\begin{align*}
    \mathbf{Y}          &: \text{model output}, \\
    \hat{\mathbf{Y}}    &: \text{target label}, \\
    C                   &: \text{all different classes}, \\
    w_{\mathbf{y}_n}    &: \text{weight}.
\end{align*}


\subsubsection{Dice and Focal}
\citeauthor{MA2021102035} have shown that a compound loss, featuring a simple unweighted sum of the Dice as well as the 
Focal loss, excells in performance in a range of 19 different (compound-) losses, for the task of pancreas segmentation.
Therefore, we have implemented this loss function a well. We use the same dice loss as in %\ref{eq:dice}
For the focal loss, we use the following equation:


\section{Results}
\section{Discussion}
\printbibliography

\end{document}
